% vim: set tw=80:
\documentclass{uniarticle}

\title{Example of a generic assignment}
\date{\today}
\author{Daniel Voogsgerd}

\bibliographystyle{IEEEtran}

% \bibliography{references.bib}

\begin{document}

\maketitle

\tableofcontents

\begin{abstract}
    \LaTeX class defines a class for \LaTeX which an optimal default of
    included packages
\end{abstract}

\section{Elements}

\subsection{Text}

We can link to other sources \href{https://google.com}{Google}

\subsubsection{Lists}

\paragraph{Fonts}

\subparagraph{Math mode}

\subparagraph{Other languages}

\foreignlanguage{dutch}{
    We kunnen ook stukken in een andere taal schrijven. Hierbij gaat babel
    wisselt babel naar de juiste taal.
}

We have equations

\begin{equation}
    E = mc^2
\end{equation}

\subparagraph{Monospace}

\subparagraph{Default}

\subparagraph{Quotes}

We have several ways of quoting.

The first would be the \verb|Verbatim|

We can also insert blockquotes using the environment

\subsection{Figures}

\begin{figure}[h]
    % \includegraphics[width=0.45\linewidth]{figures/}
    \caption{}
    \label{fig:}
\end{figure}

\subsection{}
\subsubsection{Example}

\paragraph{Test}

We can have some code

\begin{listing}
    \begin{minted}{python}
print("Hello world")
    \end{minted}
    \caption{Hello world written in Python}
    \label{listing:helloworld}
\end{listing}

\subparagraph{Testing}

This should have all headings. We can cite stuff \cite{harris_array_2020}
See code \cref{listing:helloworld}

\bibliography{References}

\end{document}
